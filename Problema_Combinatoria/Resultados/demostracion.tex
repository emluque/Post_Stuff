\documentclass{amsart}
\usepackage[utf8]{inputenc}
\usepackage[spanish]{babel}
\usepackage[T1]{fontenc}

\author{ Emiliano Martinez Luque 
}

\title{ Cuantas cadenas de tamaño $n$ pueden ser creadas con exactamente $m$ símbolos de un alfabeto de $k$ símbolos?  }
\begin{document}

\maketitle

Lo que estamos buscando es, dado $A$ un alfabeto de tamaño $k$, la cantidad de cadenas de tamaño $n$ que contienen exactamente $m$ símbolos de $A$.

Para resolver este problema primero vamos a resolver lo siguiente:

\section{ Cuantas cadenas de tamaño $n$ pueden ser creadas con todos los símbolos de un alfabeto de $k$ símbolos? }

Se propone la siguiente relación de recurrencia de $2$ variables, donde:

$n$: representa el tamaño de la cadena
$k$: representa la cantidad de símbolos en el alfabeto

\begin{equation}
f(0, k) = 0
\end{equation}
Para todo $k$, ya que la cantidad de cadenas de tamaño 0 que pueden formarse con todos los símbolos de $A$ es 0.

\begin{equation}
f(1, 1) = 1  
\end{equation}
Ya que existe solo una cadena de tamaño 1 con 1 símbolo.

\begin{equation}
f(n, k) = k( f(n-1,k) + f(n-1, k-1) )  
\end{equation}

\section{ Demostración }

Llamemos $A_{n,k}$ al conjunto de todas las cadenas de tamaño $n$ que contienen a todos los símbolos de $A$ e intentemos construir todas las cadenas que pertenecen a este conjunto.\\

Para cada símbolo $a$ de $A_{n,k}$ vamos a realizar el siguiente procedimiento:\\

1. Llamemos $A_{n-1, k}$ al conjunto de todas las cadenas de tamaño $(n-1)$ con exactamente $k$ símbolos de $A$ (o sea con todos los símbolos de $A$).\\
Para cada cadena $\alpha$ de $A_{n-1, k}$ construimos una nueva cadena concatenandole $a$, nos queda así una nueva cadena: $a\alpha$. Como $\alpha$ tiene tamaño $(n-1)$, la nueva cadena $a\alpha$ tiene tamaño $n$.\\
Ademas por definición de $A_{n-1, k}$, $\alpha$ contiene a todos los símbolos de $A$, y en particular contiene a $a$. Entonces de esta manera hemos formado todas las cadenas de $A_{n,k}$ que comienzan por $a$ y tienen al menos 2 ocurrencias de $a$.\\
Nos queda todavía construir todas las cadenas de $A_{n,k}$ que comienzan por $a$ pero tienen solo una ocurrencia de $a$.\\ 

2. Llamemos $A_{n-1, k-1}$ al conjunto de todas las cadenas de tamaño $(n-1)$ que están formadas por todos los símbolos de $A$ menos $a$.\\ 
Para cada cadena $\beta$ de $A_{n-1, k-1}$ construimos una nueva cadena $a\beta$. Se cumple entonces que $a\beta$ tiene tamaño $n$ y ademas dado que $\beta$ contiene a todos los símbolos de $A$ menos a $a$ y le hemos concatenado $a$, $a\beta$ contiene los $k$ símbolos de $A$.\\

A partir de estos dos pasos hemos construido todas las cadenas de $A_{n,k}$ que comienzan con $a$. Como por definición de $A_{n,k}$, cualquier cadena de $A_{n,k}$ necesariamente tiene que comenzar con un símbolo de $A$ y $a$ es un elemento arbitrario podemos aplicar el procedimiento para construir todas las cadenas del conjunto $A_{n,k}$.\\ 

Como por hipótesis, $|A_{n-1, k-1}| = f(n-1,k-1)$, $|A_{n-1, k}| = f(n-1,k)$ y como existen exactamente $k$ símbolos de $A$ se cumple que $f(n, k) = k( f(n-1,k) + f(n-1, k-1) )$.\\


\section{ Cuantas cadenas de tamaño $n$ pueden ser creadas con exactamente $m$ símbolos de un alfabeto de $k$ símbolos? }

Asumiendo que la función anterior es correcta, podemos proponer la siguiente función: 

\[ 
g(n,m,k) = \binom {m}{k} \times f(n,m)
\]


Llamemos $A_{m}$ al conjunto de todos los subconjuntos de $A$ con exactamente $m$ elementos. Es conocido que la cantidad de subconjuntos de tamaño $m$ de un conjunto se puede calcular como: $|A_{m}| = \binom {m}{k}$ \footnote{ http://en.wikipedia.org/wiki/Combination } . Sabemos ademas por el resultado anterior que la cantidad de cadenas de tamaño $n$ que pueden hacerse de un conjunto de tamaño $m$ es igual a $f(n,m)$, a partir de lo cual derivamos la formula.\\


\end{document}
